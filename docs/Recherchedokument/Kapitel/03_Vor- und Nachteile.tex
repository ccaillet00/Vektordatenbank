\section{Vor- und Nachteile}
Wie bei jeder Technolgie gibt es auch bei Vektordatenbanken nicht nur Vorteile sonder auch Nachteile. 

\subsection{Vorteile}
\begin{outline}
    \1 \textbf{Effiziente Ähnlichkeitssuche:} Vektordatenbanken zeichnen sich durch die Identifizierung ähnlicher Datenpunkte auf der Grundlage ihrer Vektordarstellung aus und ermöglichen Anwendungen, die eher auf Bedeutungszusammenhänge als exakten Keyword Übereinstimmungen basieren.
    \1 \textbf{Verwaltung hochdimensionaler Daten:} Vektordatenbanken sind darauf ausgelegt, grosse Mengen hochdimensionalen Daten effizient zu verarbeiten wodurch sie sich für moderne datenintensive Anwendungen eignen.
    \1 \textbf{Skalierbarkeit:} Auch bei Millionen Datensätzen liefern Vektordatenbanken performante Ergebnisse und können horizontal skaliert werden um wachsende Datenmengen und Verarbeitungsanforderungen zu bewältigen.
    \1  \textbf{Leistungsoptimierung:} Vektorisierte Algorithmen wie \emph{Hierarchical Navigable Small World} HNSW, \emph{Locality-Sensitive Hashing } LSH, \emph{Product Quantization } PQ oder \emph{Approximate Nearest Neighbors Oh Yeah} ANNOY beschleunigen Such- und Abrufvorgänge was zu schnelleren Antwortzeiten führt.
\end{outline}
\subsection{Nachteile}
\begin{outline}
    \1 \textbf{Komplexität beim Verstehen:} Wenn man mit relationalen Datenbanken arbeitet kann der Umsteif auf Vektordatenbanken zunächst komplex wirken. Es braucht etwas Einarbeitung und erfodern je nachdem ein fundiertes Verständnis mehrdimensionaler Datenstrukturen und Algorithmen.
    \1 \textbf{Datenintegrität und konsistenz:} Die Gewährleistung einer hohen Datenqualität und konsistenz über Vektoreinbettung hinweg kann kompliziert sein.
    \1 \textbf{Ressourcenintensiv:} Die Verarbeitung hochdimensionaler Vektoren kann rechenintensiv sein. 
    \1 \textbf{Abfragegenauigkeit vs. Leistung:} Es ist notwendig, die Abfragegeschwindigkeit mit der Genauigkeit und Präzision der Vektorsuchergebnisse in Einklang zu bringen, was schwierig sein kann.
\end{outline}
\footnote{Für das Kapitel 3 wurden diverse Quellen genutzt.} % Verbessern!